\documentclass[letterpaper,12pt]{article}
\usepackage[round]{natbib}
\usepackage{aas_macros,url}
\begin{document}
\title{90prime Observers Manual//Revised 04/06/2004//Revised 11/30/2004}
\author{G. Grant Williams, E. Olszewski, M. Lesser, and many more}
                                                                                
\maketitle

\section{Abstract}

90prime is a prime focus wide-field imager for the Steward Observatory 90''
telescope. The optical design includes a four element corrector and six
position filter wheel. The focal plane array is a mosaic of four Lockheed
4k x 4k CCDs.

The camera will provide an imaging area of 1.0 square degrees. The
edge-to-edge field-of-view including the inter-CCD spacing is
1.16 x 1.16 with a plate scale of 30.2''/mm or 0.45''/pixel. The
instrument is currently available for observing. 
                                                                                
\section{Overview}

This section will give a brief overview of the system and a table 
of characteristics.

The SPIE paper by Grant Williams etal, which you
should read sometime, and which you should reference
as you publish 90prime data, is available
in various formats here (link to paper)

Please note that there is ZERO 90prime staff. The folks
who built 90prime and the CCD staff and the operations staff
will help you, but we hope that you'll help us to
solve problems, and we hope you'll contribute
knowledge and computer code to make data acquisition
and analysis and pipelininging more efficient or possible.

90prime is meant to be run remotely, from the control room.

a) although the operator is repsonsible for this, the observer
should be aware that the 90prime dewar is to be filled daily.
The hold time is actually 2 days, but the balance of the
telescope is thrown off if we don't add nitrogen daily.

b) Filters are not to be changed by observers, period.
Gary Rosenbaum, Bill Wood, and some others can make the requested
filter changes. Plan ahead: Bill, for instance, doesn't work
on Fridays and Saturdays.

Many of our filters reside permanently in filter holders to
minimize handling of the actual glass. 

Gary Rosenbaum is the only person (with Bill Wood in an emergency)
authorized to put filters into filter holders and remove
filters from filter holders. Observers simply are not
authorized to do this.

c) Inside the control room, the observer needs to know
how to start up the CCD computer (BokCCD2) and how to
ensure that the CCD program is happily running.

d) Most of what the observer does happens when they are sitting
at the consle of the data computer Stjukshon (to be renamed Bok
when the old Sun computer Bok is permanently turned off
and its IP address freed up).

The observer needs to know how to start the AzCam server (CCD data
taking), how to set parameters within it, and how to set
up the dataserver daemon.

The observer needs to know how to start the 90prime Instrument GUI
program, and how to manipulate filters and focus.

The observer needs to know IRAF or some such language to make
decisions about data quality.

The observer needs to know about the autoguider routines although
actual autoguiding rests largely in the hands of the telescope
operator.

The observer needs to know how to put data on tape or DVD or CD.

e) The observer needs to know that, at this moment, the only choices
for good flatfields are twilight flats or dark-sky flats.
Our current dome flats are not adequate. There are important
consequences for getting such data in bad conditions, or in trying
to use "too many filters."

f) These are thinned CCDs, so they fringe in the red. The observer
has to take adequate data to allow them to remove fringing.
At this moment, we don't know if it's possible to build up
a library of fringe frames that will work from run to run.

This manual will discuss all of these issues in detail.

\section{Quick Facts}

2) Quick (shorthand) startup instructions for CCD computer
   and Data computer and autoguider computer

On each CCD, East is up and North is right. So if you need to move your object
left, move it north. Pixel scale is about 0.45 arcsec/pixel.

ccd3     ccd4

ccd1     ccd2


CCD3 has a trap about 2/3 of the way up, about in the middle. Data above
the trap are useless.

CCD1 has some weird pixels in upper left.

CCDs 3 and 4 have 1 or two bad columns.

There may be other features to be described later.

As of 3pm on 7Apr04, the best focus for CCD2 isn't the same as for the
other CCDs. You may have to ignore the existence of CCD2 for a while,
unfortunately. We'll let you know if we can fix focus by
tilting the dewar relative to the optics (we do this with
the focus actuators, fyi).


Since there's a substantial gutter between ccds, roughly 1000 pixels,
you'll likely want to zero your coords somewhere other than the
unseen center of the array. You might want to put it on
the center of some chip. That said, note that offsetting
by TCS keyboard commands wants offsets in hms and dms. If you
paddle over, you can read off the offsets in arcsec,

The full well of the CCDs is low, about 70k electrons, or about 35k
ADU. The last measured values are:
Amp 1: 1.78 e/DN
Amp 2: 1.81 e/DN
Amp 3: 1.68 e/DN
Amp 4: 1.67 e/DN
Amp 5: 1.71 e/DN
Amp 6: 1.72 e/DN
Amp 7: 1.62 e/DN
Amp 8: 1.71 e/DN

\section{Start-Up Instructions}

90prime gui (on the linux box STJUKSHON) startup/usage instructions


1) There is a set of mounting and testing instructions that
   in principle exists. So this manual starts from the
   successful mounting of the instrument.

2) Filters- Only Gary Rosenbaum is authorized to put filters
   into filter holders. I reiterate, the only person to touch
   raw glass is Gary Rosenbaum. the only person to clean filters
   is Gary Rosenbaum. The only person to remove filters from
   filter holders is Gary rosenbaum.

   Got it? Only Gary Rosenbaum will handle filters. Plan ahead.

   We now have a whole bunch of 6" holders, into which the existing
   90prime-owned filters will be permanently mounted. We need to
   make a holder-box for the filters-plus-holders. 

   Once filters are permanently mounted, Bill Wood will be allowed
   to remove holders from the instrument and install new ones.
   Things should be relatively safe once filters are permanently
   mounted. Observers CANNOT add/remove filters from the
   instrument even after they're mounted in holders.

3) Nitrogen filling- The operator will fill the dewar once per day.
   Hold time is longer than 2 nights, but the change in weight
   as nitrogen boils off will unbalance the telescope if it
   isn't filled daily.

4) Starting the 90prime Instrument gui and running the gui

      Double click on the icon "90prime Instrument GUI"
      The control interface will come up running (in older versions
      you had to make it run).

      You'll see a bunch of lights and 4 tabs- Focus, Filter Wheel,
      Guide Camera, and Misc.

      For now, you need to type the follwoing command into a xterm
      window, "dataserver". This command now works as of 7Apr04, but
      apparently one can go back to the old way if it fails. It
      will require a phonecall to go back to the old way.
      This command sets up the way we transfer data from the CCD computer
      (running WINDOWS) to STJUKSHON (this computer, running LINUX).
      It also sets up a DS9 display for quick look. You'll
      want to use "mscdisplay" for real analysis like "imexamine".

\section{Instrument Control}

      We'll now quickly go through the tabs:

       THE FOCUS TAB

         On startup, you'll see three LVDT readings, one for each
         focus actuator. Usually you'll see the last digit change
         by one. That's insurance that things are actually running.
         The LVDTs give an absolute reference frame.

         Now we'll explain the buttons one by one:
        
       a)  Focus- this actively moves the three focus actuators
          a predefined amount. the predefined amount is set by typing
          a number of encoder steps (as opposed to LVDT increments)
          into the steps windows. If the "link" button is on, you only
          type into B, and that same number appears in A and C.
          That's the mode you'll use.

          Do hit a carriage return.

          The gui also converts encoder steps into microns, and keeps
          track of total steps.

          Actual focussing is done using the CCD control software.
          The software will take an exposure, close the shutter,
          shift charge, change the focus, opern the shutter, etc,
          and then finally readout. 

          We have aligned the focal plane of the CCDs as best we can with
          the focal plane of the optics. If you move actuators A B C
          different amounts, you've destroyed this work.

          For reference, in January 2004, good focus was A=2.237, B=2.217,
          C=2.309. While the value will change (It should go down xxx
          as the temperature goes up), the offsets between the three actuators
          should not change. So check that these offsets are about right,
          and if they aren't, you should look for a note or you should make
          a phonecall to see if there was an update or if the previous
          observer screwed up.

          You can keep track of the focus by clicking on the button
          SET REFERENCES.

       b) SET REFERENCES
    
          We jsut discussed it above.

       c) TWEAK FOCUS

          The instrument flexes. That's bad. The LVDTs allow us to
          monitor that flexure. TWEAK FOCUS unflexes the telescope by
          driving the LVDTs back to their reference values. So after
          a focus run, do click on SET REFERENCES.

          To reiterate, you've just gotten into perfect focus. We'd
          like to believe that if you move to somewhere else in the sky,
          all that changes is because of instrument flexure. To first
          order, we can remove that change by hitting TWEAK FOCUS.
          Of course, the telescope itself flexes too, and perhaps
          the optics assembly itself flexes. At this point, we are
          pretty convinced that most problems you'll encounter
          will be fixed by TWEAK FOCUS and/or a focus run.

          DO NOT change individual actuators. It's currently
          a real pain to align the focal plane. If you want to
          help us write software to align the focal plane quickly,
          we'd be grateful for your help.

       d) ZERO TOTALS

          It zeros out the totals windows.

       e) ZERO ENCODERS

          It zeros the encoder windows. 
 

       THE FILTER WHEEL TAB

       Since this part of the gui was written, our way of intereacting
       with it has been modified to make it better.
   
       So ,let's start by clicking on CONTROL. 
       That will bring up a popup that you'll leave running at all times.
       So move the popup to a favorite place.

       Many of the commands in the main tab are repeated in the popup.
       Ignore the ones in the main tab, use the popup.

       Therefore, aside from some reassuring lights, here is a list
       of the only buttons you'll push in the main FILTER WHEEL tab:
        CONTROL (we've already talked about the fact that it brings
         up a super useful popup)
        POPULATE WINDOW - this allows Gary Rosenbaum to put filters
         into the instrument.
        ROTATION OFF (and ON)- this might be used in an emergency, as
         would be explained to you by one of us.

       So, the rest of this section will explain the "90prime Filter
        Wheel Control" popup:

        INIT FILTER WHEEL- this is done right after Gary Rosenbaum
         loads filters, or after a power failure. Basically, the filter wheel
         is spun around once, stopping momentarily to read the tabs
         in each filter holder. The computer then knows the filter
         number in each filter position (filter number refers to the filter
         holder whereas filter position is one of six slots in the filter
         wheel).

        READ FILTERS- There's a file on this computer, at
         /home/bokobs/90prime/galil/filters.txt
         that contains the transformation between filter position
         and filter name. This file is edited by the Gary Rosenbaum.
         Hitting READ FILTERS applies this program. Now
         you know the real names of your filters.

        SLIDE- our filter wheel works by rotating into position, then
         sliding the filter in. So we CHANGE FILTERS, then SLIDE.
         SLIDING is controlled with an in/out toggle. When a filter
         is in the beam, the CHANGE FILTERS button is disabled so
         that you can't rotate with a filter in the beam.

        now CHANGE FILTERS- assuming you have no filter in the beam,
         you can set the filter name in the window, and then hit
         CHANGE FILTERS

        CLOSE closes this popup

        STOP RUNNING 
        You'll almost never be instructed to push this button



    THE GUIDE CAMERA TAB

      Focussing is now done at the TCS console by the operator.
      You'll have to INITIALIZE and CHANGE guide FILTER (guide filter
       is different from science filter, guider light is picked
       off before science filter)

      We've provided a few 1" round filters for now in the guider:
        1) blank/clear
        2) RED
        3) neutral density
        4) blank
        5) green
        6) blue

      We don't have much experience for how much guiding with the
      'correct filter' matters.


     The guide camera is at a fixed location. It's fed by a pickoff
     mirror. If you don't have guide stars, your only choices are
     to increase the guider exposure time or to move the telescope a bit.
     The guider field is roughly 4x4 arcmin. The telescope field is roughly
     60x60 arcmin, with roughly 10 arcmin gutters between CCDs.

     From pixel (3993,3868), roughly, on CCD1 (lower left CCD in
     the mosaic), one offsets 328 arcsec WEST and 3123 arcsec NORTH
     to center that star on the guider. This is just a useful
     fact.


     MISC.

     At this point, these are potential commands that'll only
     be used by you at Grant's command.

\section{CCD Control}

     The CCD (on rack mounted BOKCCD2) computer

     This is a machine running WINDOWS. It talks to the CCD
     controller. While it is in principle a stand-alone
     machine, we use it to run the data-taking gui and to
     ship data to STJUKSHON.

     Mike Lesser has a prototype version of the LABVIEW executable
     (LABVIEW is what makes the guis for us) that will run in linux,
     thus allowing the gui to appear on stjukshon, thus making
     it easier for one person to observe. This version is not totally
     tested. We'll let you know when it appears.



     If you turn the machine or or reboot it, the 90PrimeAzCamServer
     will come up automatically. Yuo can minimize it.

     Then double click on AzCamTool icon in upper right. It'll
     load a bunch of stuff and be running. This is what you need
     to take data.

     You'll need to set some parameters (ps: there's a longer version
     on the 90prime site at compton.as.arizona.edu/90prime)

     1) Click on Detector. It'll say 4096x4096, binned 1x1, with 40 cols
        of overscan per ccd. You'll not have to change this, so
        hit cancel (clearly if you do make changes, hit 'apply")

     2) RESET button resets the controller. It might sometime be
        needed.

     3) PREFERENCES has a bunch of flags. The only one you might
        want to check that's not already checked is "Prompt
        for Image Title"

     4) click on FILENAME button.
        Here's where you set where on STJUKSHON the data will end up.
        You'll also control rootnames and sequence numbers and several
        automatic aspects of naming here.

     5) Save Image should always be on

     6) You can type into anything yellow. Be careful not to mess with
        binning, you could waste a lot of time if you change that
        and forget.

     7) there's a MODE information window:
        you'll see IDLE, PREPARING, EXPOSING, READOUT, CREATING.
        The last one means that the readout stream is being properly
         assembled and shipped.

     8) There are many other features: Imagetype, expose/pause,
        sequence, testimage, title, exposure time. With luck these
        will quickly be self evident.

     9) Lots of other abilities I haven't described (for instance
        RUN SCRIPT), partly because these abilities are new
        and/or not completely tested, or because edo doesn't
        yet know how they work. 

\section{Data Archiving}

The DDS4 dat drive is called /dev/nst0, or /dev/nst0m for compressed.
If you already have compressed data, this compression will
buy you little/nothing. 

There's a dvd reader/writer. You can use DVDRECORD or XCDROAST.
Neil Lauver is the only one we know who's successfully written
DVDs. We'll add more later.

There's also scp to downtown or to your laptop.

There's also a USB option, plug it in. The mountpoint is
/mnt/diskonkey
so say cp *.fits.gz .
or some such command, but you know that.

STEWARD MAKES NO AUTOMATIC ARCHIVES OF ANY SORT. IT'S UP TO YOU.
[we have chatted with noao about becoming part of their
automatic archive, but this is far from coming to pass.]

\section{Calibration Images}

Flatfields, fringing and the like

When we conceived of 90prime, we assumed flats would be twilight
or dark sky flats. We are slowly trying to make working dome flats
since the weather often won't cooperate.

Note that the operators work from 0.5 hours after sunset to
0.5 hours before sunrise. This is not ideal for flats. 
With other instruments, the observer makes a deal with
the operator to come early and go home early. If taking only
eveing flats, say, is insufficient, write a polite note
in the trouble report. Bob Peterson and Paul Smith and
Bill Wood are aware that some changes may need to be made.
Exactly what those changes could be is unclear.

As of April 7, 2004, the flatfield lamps are still uneven
with specular reflections. My guess is that right now they're
not useful. We're working SLOWLY on the problem.

There are fringes in the redder passbands. You need to take
the proper data to make fringe frames. We'll see if we can
make a library of fringe frames.

QE curve is found at
\url{adansonia.as.arizona.edu/~edo/90prime1_qe.pdf}

CCD layout is found at
\url{http://compton.as.arizona.edu/90prime/docs/CCDconfig.ps}

\section{Filters}

Windhorst filter curves are found at
http://www.noao.edu/noao/mosaic/filters.html
click on "click here"
these are 5.75" square filters
some vignetting occurs

90prime owns 13 6" square filters
Bessell U B V R I
SDSS u' g' r' i' z'
Washington C M DDO51
Note that Olszewski bought 4, Olszewski/Bechtold bought
1 or 2, NAU bought1, 90" bought 1, Xiaohui Fan bought 4.

\section{Temperture Measurements}

There are two thermocouples inside the 90prime topbox.  The values in
unknown units are currently displayed in the ``Misc'' tab on the 
instrument GUI.

There are also four Sensatronix TempTrax probes located at various 
locations throughout the dome.  A web page displaying the current 
temperatures is at \url{http://boktt1.as.arizona.edu/}.

The outside/dome temperature is displayed on the east control room wall
with an analog indicator.  There are also various weather stations
on Kitt Peak for this purpose.

\section{October Notes}


xxxxxxxx

3) CCD information

There are four Fairchild 485 CCDs in the dewar, 4096x4096 15 micron
pixels each, with approximately 1000 pixel gaps between CCDs.

Each CCD is read out through 2 amps, giving 8 total channels,
which are stored as a single FITS multi-extension file.
At this moment, we urge you to consult the NOAO IRAF MSCRED
pages/manuals, and to say "help mscguide" within IRAF
for how best to manipulate/reduce these data.

Readout time (time to put the data in memory on
the CCD computer) is 50 seconds. Additional time of about 30 seconds is
needed to de-interlace the data, to create fits, and to place it
on the data computer. So the total "readout time" is about
90 seconds. We are exploring ways to drop this time to about
70 seconds, but can think of no way to go faster except to
buy more channels for the CCD electronics.

The approximate scale is 0.45 arcsec/pixel, though at this point
the astronomer should check this number themselves.

The dewar is filled once daily, although the hold time
is about 2 days. Enough nitrogren boils off in one day
to affect the balance of the telescope.

Mike Lesser has set up the CCDs in the lab. He notes
the following characteristics (but astronomers are encouraged
to derive any of these numbers for themselves, see the link HERE):

a) photons/data-number is approximately 1.7. A file is
available HERE (and will soon be in the data header as well)
to see the numbers for each of the 8 amps.

b) Readnoise is about 12 electrons. As above, the
individual numbers for each amp are available HERE.

c) Saturation occurs at about 70k ELECTRONS.
   Actually, Chip4 saturates first, at about 60k electrons,
   so to be safe, keep your data below about 33,000 data numbers.
   (there is also a "zero level", zero plus overscan, to remember)

   You can determine for yourself if you can go above 33k DN
   if you ignore chip 4. 

d) data are linear up to saturation.

e) At the moment, we give you 20 columns of overscan
per amp. This number is easy to change, but all the consequences
of it (keywords and such) have never been tested.

f) Dewar temperature should be -187; camera temperature should
be -102.

g) Dark current is small, and was unmeasurable in the lab.


Other CCD issues

a) At the moment you cannot read out just one chip (note that
the readout time for one chip is the same as for 4). If
one-chip-read is deemed to be popular, we'll have to create special
code to do that.

b) At the moment, we do not allow region of interest readout.
There are interesting consequences for reagion-of-interest in 
multi-chip, multi-amp readouts. Mike Lesser can explain
it in more detail.

c) Yes, you can bin, it's easy to do. But all of the
display keywords will be wrong at this moment.

d) The shutter can be seen in the following DOCUMENT. It is
designed to give the identical exposure time across the field,
there is no shutter equation (but if your data depend
critically on it, talk to us). DO NOT TAKE JILLIONS
OF SHUTTER TIMING TESTS. Minimum shutter time is of order
0.1 seconds, maybe shorter.

e) There is crosstalk between amps on an individual CCD.
So you'll see a ghost image of a bright star on
the other half of the CCD. There are routines for removing
crosstalk. We do not have it well characterized at this point.



4) Filters

We reiterate what is said in section 1: NO OBSERVER MAY
CHANGE FILTERS. PLAN AHEAD. ONLY Gary Rosenbaum may
put filters into filter holders.

90prime uses 6" filters (6.0 x 6.0 inch square filters), 12mm
in thickness, 5.75" active area. This is slightly larger
than the NOAO mosaics. 

Anyone wanting to use smaller filters should see us, WELL IN ADVANCE,
for the autocad drawings for them to make filter inserts. Given
the large gap between CCDS, on must think carefully about
the positioning of undersized filters.

We currently own the following xx 6" filters, which live
permanently at 90prime, in their own named filter holders:

a) UBVRI (the Bessell prescription, with "Cousins" R and I of course)
b) SDSS u' g' r' i' z' 
c) Washington C M DDO51

[Note that Xiaohui Fan bought 4 of these filters, Olszewski
bought 4, Olszewski and Bechtold bought 1, NAU bought 1, Operations
bought 1, and the money for 2 came from xxxx]

In addition, Rogier Windhorst has loaned his set of XX BATC filters (LINK),
which are 5.75" in size. These filters live in their protective
boxes and are loaded into special 5.75" holders when needed. PLAN AHEAD.
In addition, Windhorst can choose to use the filters at some other
telescope (NOAO, Palomar, etc), so if you choose to use them,
checking with him in advance for conflicts is imperative.

Finally, there perhaps may be filters owned by individual observers
that you can borrow. 



5) Archiving data

Stjukson contains provisions for the following archiving possibilities.
Note that we DO NOT "save the bits" as an institution. It should
be the case that your data will not need to be removed right away,
but on a long 90prime run data from the beginning of the run may
need to be removed. So check your archival media at home as soon
as possible. 

There is a DAT4 drive, called /dev/st0 (regular, non-compressed)
or /dev/st0m (hardware compression). It of course senses whether
you've put a DAT4, DAT3, DAT2, or DAT1 tape in it. YOU MUST
PROVIDE YOUR OWN TAPES.

There is a DVD drive. You access the drive using the XCDROAST
program, which has a gui interface. Make sure you know how to use
XCDROAST in advance of your run. 

The drive allows the use of the following sorts of DVDr's
xxxxxxx

There is also a CD writer, but it's defunct now that stjukshon
is in the rack. We'll be buying one that allows sideways mounting.

There is also a connection for a Firewire external drive and
a USB external drive. Again, now that it's raqck mounted, getting
to the connections is hard. Jeff Rill will be providing a junction
box. 

6) Computers

BokCCD2 (the CCD computer) is a Windows PC that controls
the CCD controller. Astronomers are not to use it beyond
what's needed for observing (when all is well, you never
even look at that computer).

Stjukshon has a RAID array (software RAID) with 5 60Gig
SCSI disks... 282 Gigs useful size.

7) How to start up the needed computers and GUIs

BokCCD2 boots up running the 90prime server. As long
as it's running, you're good. If you get an error
message talking about AzCam vi's, see if BokCCD2
has hung. If you can move the cursor, you can try killing
the 90prime server and restarting it with the icon.
That often doesn't work, and you need to reboot using the
START button on the lower left. Even more extreme is a hard
reboot, done by holding in the power switch for 5 seconds till
the computer powers down, then restarting it.

Stjukshon, the Linuc PC running under Fedora, should always
be running, and should not be rebooted by anyone other
than the Computer Support Group. Besides IRAF, you need the
following:

a) 90prime instrument gui (icon on left). This controls 90prime
itself.

b) AzCam tool gui (icon on left). This controls the CCD data
taking.

c) In any terminal window type "dataserver". This brings up
a ds9 window and establishes communication with BokCCD2.
[You can then minimize that terminal window]

More information about USING these GUIs will be found below.

In addition, communication to the operator's TCS computer
(telescope control software) is done through the autoguider
computer on the operator's console. Make sure that "telcom
server" is running. The operator can set up the guider
for you.


8) Short cheat sheet on AZCAM GUI

The purpose of this section is to remind you what to do,
not to tell you in great detail how to do it.

a) In a regular window, mkdir the directory to which you want data sent.

   Then click FILENAME on the gui. It'll allow you
   to pick that directory, so that BokCCD2 can write into that
   directory on stjukshon. It'll also allow you
   to set up a root filename, add a sequence number to that name,
   and other naming things. You can also check the generic
   "test.fits" choice, and your data will be written (and overwritten)
   into that file.

b) Anything in YELLOW is changeaqble by an astronomer.
   Basically, you're concerned about Object and Exposure,
   and Title.

   Then you press EXPOSE and off you go.

c) During an exposure, hitting EXPOSE button again invokes
   a PAUSE sequence. Several options are available.

d) One of the choices for OBJECT is FOCUS. Choosing imagetype
   of FOCUS, then hitting EXPOSE, will invoke a special
   FOCUS POPUP which allows you to set those extra parameters.

e) You can also click on SEQUENCE (just to the right of FILENAME)
   for a simple observing sequence of, say, 3 identical exposures.
   [We do not yet have the equivalent of the famous 4shooter
    "dinner" command that allowed multiple sequences in different
    filters, and with differing exposure times, to be invoked.]

f) There are other useful buttons on the left side. One that
   you should know about is "INSTRUMENT". In that popup you'll
   note that there exists a filename for translation of filter
   number to filter name IN THE HEADER of the FITS file.
   "filters.txt" is the default.

g) One aspect of this GUI that is seemingly weird isw that you need to
   close any popups before exposing. Trust us on this one.  This
   differs from the other gui! [Don't ask why, it's a subtle bug of
   some sort.]

9) Cheat Sheet on 90prime Instrument GUI

   There are 4 main tabs: Focus, Filter Wheel, Guide Camera, Misc.

   Let's start with Filter Wheel Tab:
   a) Whenever filters have been changed, you need to edit
      /home/primfocus/90prime/galil/filters.txt
      This file, the same one used by the AzCam program, is
      used to allow the astronomer to see real filter names
      rather than filter numbers. [We note that filter
      numbers are encoded on the filter holders themselves
      and sensed by the software.]

   b) Hit the CONTROL button. It'll bring up a pop-up.
      You can leave this popup open forever.

   b) On power-up, you'll need to do the following in this tab:
      Initialize Filters (this allows the software to read filter numbers)

      Read filters (this tells the software the correlation between
       filter number and real filter name)

      Once you've done both of these steps, the red bar will be green
      with a filter name in it. This means that that particular
      filter is ready to be slid into the beam.

      We remind you here that choosing a filter is divided into
      two steps: rotating the wheel, and then sliding the filter
      into the beam. At some point these will be combined into
      one step.

   c) Rotating and sliding
      There are two buttons: ROTATE and SLIDE

     ROTATE is self explanatory. Watch for the little centered pop-up.

     SLIDE needs one explanation. The bar is GREEN when it's
     safe to slide in, ORANGE while the filter is sliding, and
     YELLOW when the filter is in the beam. After initializing,
     described above, any RED bar is BAD.

     If you get RED while sliding in, slide out and try again.
     If you still get RED, call Gary Rosenbaum.

 
     Now the FOCUS tab

     Start by clicking the NOMINAL PLANE button. This ensures
     that actuators A B C are set to their nominal differences,
     ie, that the CCD focal plane is perpendicular to the beam.

     Do a focus sequence. Derive the best focus. Your focus
     sequence is in STEPPER MOTOR STEPS, so you know how many
     stepper motor steps to move the focus to get best focus.
     [It's a relative number from where you are, NOT an absolute
      number.]

     Type that number in to STEPS A B C (if the LINK button
     is on, typing into B and a carriage return gives you
     A B C identically). Then hit the FOCUS button.

     So presumably,you're at good focus with the optimal
     differences between A B C maintained. Hit the SET REFERENCES
     button to remember this value.

     As the telescope moves in the sky, the instrument flexes.
     So the stepper motors flex, and you'll never know!
     We added absolute encoders, called LVDTs, to sense
     this flexure. So if your LVDT readings are different
     from what you saved by setting references, hit RESTORE
     FOCUS to unflex the instrument. RESTORE FOCUS drives
     the LVDTs to their correct value, thus overcoming the
     flexed stepper motors.


     So a typical focus and observe routine might be:
     NOMINAL PLANE, followed by a focus routine with AzCam server,
     followed by a setting of best focus using the FOCUS key,
     followed by SET REFERENCES,follwed by observing and 
     RESTORE FOCUS. We recommend RESTORE FOCUS before each
     exposure begins.


The Autoguider tab
xxxx

The MISC tab
xxxx

\end{document}
